%%%%%%%%%%%%%%%%%%%%%%%%%%%%%%%%%%%%%%%%%%%%%%%%%%%%%%%%%%%%%%%%%%%%%%%%%%%%%%
%%%%%%%%%%%%%%%%%%%%%%%%%%%%%%%%%%%%%%%%%%%%%%%%%%%%%%%%%%%%%%%%%%%%%%%%%%%%%%
%%
%% Dokumentace k projektu č 2 do předmětu IZP. Tento soubor je upravená
%% verze ukázkové dokumentace.
%%
%% Autor: Vlastimil Slinták <xslint01@stud.feec.vutbr.cz>
%% Verze: 1.0
%% Datum: po 22. listopadu 2010 14:16:40 UTC
%% Popisovany soubor: proj2.c
%%%%%%%%%%%%%%%%%%%%%%%%%%%%%%%%%%%%%%%%%%%%%%%%%%%%%%%%%%%%%%%%%%%%%%%%%%%%%%
%%%%%%%%%%%%%%%%%%%%%%%%%%%%%%%%%%%%%%%%%%%%%%%%%%%%%%%%%%%%%%%%%%%%%%%%%%%%%%
\documentclass[12pt,a4paper,titlepage,final]{article}
	% Pri pouziti XeLaTeX pouzij unicode fonty
	\usepackage{fontspec}
	\usepackage{xunicode}
	\usepackage{xltxtra}
	\usepackage{amsfonts}		% Kvuli matematickym fontum
	\usepackage{mathtools}

	% cestina a fonty
	\usepackage[czech]{babel}
%	\usepackage[utf8]{inputenc}
	% balicky pro odkazy
	\usepackage[bookmarksopen,colorlinks,plainpages=false,urlcolor=blue,unicode]{hyperref}
	\usepackage{url}
	% obrazky
	\usepackage{graphics}
	% velikost stranky
	\usepackage[top=3.5cm, left=2.5cm, text={17cm, 24cm}, ignorefoot]{geometry}

	\usepackage{hyperref}		% Hypertextove odkazy v PDF
	\hypersetup{%
		backref,
		colorlinks=true,
		pdftitle={IZP: Projekt 2},
		pdfauthor={Vlastimil Slinták <xslint01@stud.feec.vutbr.cz>},
		pdfkeywords={IZP, projekt, iterace, rekurze, tanh, logax, wam, wqm},
		pdfsubject={Projekt 2 do předmětu IZP - iterační výpočty}
	}

	% -- Vkladani obrazku
	\newcommand{\fig}[3]{%
		\begin{figure}[ht]
			\centering\includegraphics[height=#3, keepaspectratio]{#1}%
			\caption{#2}%
		\end{figure}
	}

\begin{document}

%%%%%%%%%%%%%%%%%%%%%%%%%%%%%%%%%%%%%%%%%%%%%%%%%%%%%%%%%%%%%%%%%%%%%%%%%%%%%%
% titulní strana
\def\author{Vlastimil Slinták}
\def\email{xslint01@stud.feec.vutbr.cz}
\def\projname{Iterační výpočty}
%%%%%%%%%%%%%%%%%%%%%%%%%%%%%%%%%%%%%%%%%%%%%%%%%%%%%%%%%%%%%%%%%%%%%%%%%%%%%%

\begin{titlepage}

% \vspace*{1cm}
\begin{figure}[!h]
  \centering
  \includegraphics[height=5cm]{img/logo.eps}
\end{figure}

\vfill

\begin{center}
\begin{Large}
Dokumentace k projektu pro předmět IZP\\
\end{Large}
\bigskip
\begin{Huge}
\projname\\
\end{Huge}
\begin{large}
projekt č. 2
\end{large}
\end{center}

\vfill

\begin{center}
\begin{Large}
\today
\end{Large}
\end{center}

\vfill

\begin{flushleft}
\begin{large}
\begin{tabular}{ll}
Autor: & \author, \url{\email} \\
 & Fakulta elektrotechniky a komunikačních technologií \\
 & Vysoké Učení Technické v~Brně \\
\end{tabular}
\end{large}
\end{flushleft}
\end{titlepage}


%%%%%%%%%%%%%%%%%%%%%%%%%%%%%%%%%%%%%%%%%%%%%%%%%%%%%%%%%%%%%%%%%%%%%%%%%%%%%%
% obsah
\pagestyle{plain}
\pagenumbering{roman}
\setcounter{page}{1}
\tableofcontents

%%%%%%%%%%%%%%%%%%%%%%%%%%%%%%%%%%%%%%%%%%%%%%%%%%%%%%%%%%%%%%%%%%%%%%%%%%%%%%
% textova zprava
\newpage
\pagestyle{plain}
\pagenumbering{arabic}
\setcounter{page}{1}

%%%%%%%%%%%%%%%%%%%%%%%%%%%%%%%%%%%%%%%%%%%%%%%%%%%%%%%%%%%%%%%%%%%%%%%%%%%%%%
\section{Úvod} \label{uvod}
%%%%%%%%%%%%%%%%%%%%%%%%%%%%%%%%%%%%%%%%%%%%%%%%%%%%%%%%%%%%%%%%%%%%%%%%%%%%%%

Tento dokument popisuje implementaci matematických funkcí \texttt{tanh} --
hyperbolický tangens, \texttt{log} -- obecný logaritmus a dvou statistický
funkcí -- aritmetický a kvadratický vážený průměr.

Implementace prvních dvou funkcí nevyužívá standardní knihovnu \texttt{math.h},
ani složitější datové typy. Statistické funkce již využívají matematickou
knihovnu i jeden vlastní datový typ. V dalších sekcích tohoto dokumentu
se podrobně podíváme na implementace všech čtyř funkcí.

%%%%%%%%%%%%%%%%%%%%%%%%%%%%%%%%%%%%%%%%%%%%%%%%%%%%%%%%%%%%%%%%%%%%%%%%%%%%%%
\section{Analýza problému a princip jeho řešení} \label{analyza}
%%%%%%%%%%%%%%%%%%%%%%%%%%%%%%%%%%%%%%%%%%%%%%%%%%%%%%%%%%%%%%%%%%%%%%%%%%%%%%
Pro úspěšnou implementaci zadání je nutná znalost chování zde uváděných
matematických funkcí. V této části se proto podíváme na jednotlivé funkce
a popíšeme si jejich chování.

%=============================================================================
\subsection{Hyperbolický tangens}
  Graf funkce \texttt{tanh} je na obrázku číslo \ref{grf-tanh}. Je zřejmé, že
  argumentem funkce mohou být všechna reálná čísla. Grafem je lichá funkce,
  která se pro rostoucí argument \texttt{x} asymptoticky blíží $1$, resp. $-1$.
  O této funkci tedy platí následující tvrzení:

  \[
  f(-x) = -f(x); \\
  \lim_{x \to \infty} \tanh x = 1; \\
  \lim_{x \to -\infty} \tanh x = -1; \\
  x \in \mathbb{R}; \\
  \]

  \fig{img/tanh.eps}{Graf funkce \texttt{tanh(x)}\label{grf-tanh}}{7cm}

  Výpočet lze realizovat Taylorovou řadou, která ale pro \texttt{tanh} obsahuje
  Bernoulliho čísla. Vhodnější implementace se jeví přes \texttt{sinh} a \texttt{cosh}:

  \[
  \tanh x = \frac{\sinh x}{\cosh x} = \frac{ \sum_{n=0}^{\infty} \frac{x^{2n+1}}{(2n + 1)!} }{ \sum_{n=0}^{\infty} \frac{x^{2n}}{(2n)!} }
  \]
  
  Pro tento výpočet je tedy potřeba implementovat funkce \texttt{sinh} a \texttt{cosh},
  případně provést výpočet obou funkcí během jednoho vyklu. Já jsem nakonec zvolil
  jiný způsob výpočtu -- přes exponenciální funkci:

  \[
  \tanh x = \frac{e^{2x} - 1}{e^{2x} + 1}
  \]

  Exponenciální funkce lze spočítat pomocí Taylorovi řady následovně:
  
  \[
  e^x = \sum_{x=0}^{\infty} \frac{x^n}{n!}
  \]
  

%=============================================================================
\subsection{Obecný logaritmus} \label{kap-log-mat}

  Obecný logaritmus je definován pouze pro kladná reálná čísla kromě nuly a to
  jak pro argument, tak pro základ. Pro $\log_a(x)$ platí:

  \[
  a, x \in \mathbb R^+;
  f(x) \in \mathbb R
  \]

  Graf této funkce je na obrázku \ref{logax-img}. Klesající křivka je logaritmus
  o základu $0.5$, zatímco rostoucí je dekadický logaritmus, tedy o základu
  $10$.

  \fig{img/logax.eps}{Grafy funkcí $\log_{10}(x)$ a $\log_{0.5}(x)$.\label{logax-img}}{7cm}

  Dále platí následující výroky:

  \[
  \log_a x = \frac{\log_b x}{\log_b a}
  \]
  \[
  \log_a x = -\log_{\frac{1}{a}} x,\;\;\; a \in (0,1)
  \]
  \[
  \log_a x = -\log_a \frac{1}{x},\;\;\; x \in (0,1) 
  \]

  Využití těchto dvou rovnic bude vysvětleno v kapitole \ref{kap-logax} věnující
  se implementaci obecného logaritmu.

%=============================================================================
\subsection{Statistické funkce}

V této kapitole jsou uvedeny vzorce pro vážený aritmetický a kvadratocký průměr.
Popis implementace je pak v kapitole \ref{kap-stat}.

V obou níže uvedených vzorcích platí --- $x_i$ je i-tý prvek průměru a $w_i$ je
jeho váha.

\paragraph{Vážený aritmetický průměr}

\[
\bar{x} = \frac{\sum^k_{i=1} x_i w_i}{\sum^k_{i=1} w_i}
\]

\paragraph{Vážený kvadratický průměr}

\[
\bar{x} = \sqrt{\frac{\sum^k_{i=1} x^2_i w_i}{\sum^k_{i=1} w_i}}
\]

%%%%%%%%%%%%%%%%%%%%%%%%%%%%%%%%%%%%%%%%%%%%%%%%%%%%%%%%%%%%%%%%%%%%%%%%%%%%%%
\section{Návrh řešení problému} \label{navrh}
%%%%%%%%%%%%%%%%%%%%%%%%%%%%%%%%%%%%%%%%%%%%%%%%%%%%%%%%%%%%%%%%%%%%%%%%%%%%%%


%=============================================================================
\subsection{Analýza vstupních dat}

Na vstupu od uživatele očekáváme pouze desetinná čísla ve tvaru například
$3.4$, $3e-4$, případně speciální konstantu $inf$, která představuje
nekonečno. Jakýkoliv jiný vstup bude programem vyhodnocen jako chybný a
uživateli to bude oznámeno chybovým hlášením.

%=============================================================================
\subsection{Specifikace testů} \label{testy}


\paragraph{Test 1:} Chybný vstup $\longrightarrow$ Detekce chyby.
\begin{verbatim}
./proj2 --tanh 5 <<< "bla"
./proj2 --tanh 5 <<< "nan"
./proj2 --tanh 5 <<< "0,4"

./proj2 --logax 5 10 <<< "bla"
./proj2 --logax 5 10 <<< "nan"
./proj2 --logax 5 10 <<< "0,1"

./proj2 --wam <<< "bla"
./proj2 --wam <<< "4.2"
./proj2 --wqm <<< "4.2 bla"
./proj2 --wqm <<< "4,2 nan"
\end{verbatim} 

\paragraph{Test 3:} Vstup mimo povolený rozsah hodnot $\longrightarrow$ Detekce
chyby.
\begin{verbatim}
./proj2 --logax 10 10 <<< "-3.0"
./proj2 --logax 10 -1 <<< "3.0"
./proj2 --logax 10 0 <<< "3.0"
./proj2 --logax 10 2 <<< "0"
\end{verbatim} 

\paragraph{Test 4:} Správný vstup $\longrightarrow$ Výsledek

\vspace{1em}\begin{tabular}{ll} % ll = 2 sloupce zarovnane: left,left
vstup & očekávaný výstup \\
\hline
\verb|./proj2 --tanh 5 <<< "0.5"| & 4.6211715726e-01 \\
\verb|./proj2 --tanh 5 <<< "-100"| & -1.0000000000e+00 \\
\verb|./proj2 --logax 5 2 <<< "16"| & 4.0000000000e+00 \\
\verb|./proj2 --wam <<< \| & 1.5000000000e+00 \\
  "1.5 1.0 2.5 2.0 3.5 3.0" & 2.1666666667e+00 \\
& 2.8333333333e+00 \\
\end{tabular}


%%%%%%%%%%%%%%%%%%%%%%%%%%%%%%%%%%%%%%%%%%%%%%%%%%%%%%%%%%%%%%%%%%%%%%%%%%%%%%
\section{Popis řešení} \label{reseni}
%%%%%%%%%%%%%%%%%%%%%%%%%%%%%%%%%%%%%%%%%%%%%%%%%%%%%%%%%%%%%%%%%%%%%%%%%%%%%%

Vlastní implementace jednotlivých funkcí zohledňuje jejich chování, které
bylo popsáno v kapitole \ref{analyza}.

%=============================================================================
\subsection{Ovládání programu}

Program je ovládán z terminálu a jeho výstup je čistě textový. Veškerá uživatelská
data jsou načítána ze standardního vstupu a jsou zapisována na standardní výstup.
Chybová hlášení se tisknou na standardní chybový výstup. V případě výskytu chyby
program končí a vrací \texttt{1}, jinak \texttt{0}. Program tak může uživatel
spouštět interaktivně i neinteraktivně (například ve skriptech).\newline

\noindent
Argumenty programu jsou:

\begin{tabular}{r||l} \hline \hline
	Argument & Popis \\ \hline \hline
	$-$h		 & Vypíše nápovědu a skončí. \\
	$--$tanh sigdig N	 & Vypočítá tanh s přesností na \texttt{sigdig} desetinných míst. \\
	$--$logax sigdig a & Vypočítá logaritmus o základu 'a' s přesností na \texttt{sigdig}. \\
	$--$wam		 & Průběžně počítá vážený aritmetický průmer. \\
	$--$wqm		 & Průběžně počítá vážený kvadratocký průměr. \\
\end{tabular}

%=============================================================================
\subsection{Implementace \texttt{tanh}}

  Implementace využívá poznatků z kapitoly \ref{analyza}. Využívám faktu, že
  je funkce lichá, a proto není potřeba nijak speciálně řešit záporné argumenty
  funkce. 

  Jak již bylo popsáno výše, moje implementace nevyužívá funkcí \texttt{sinh} ani
  \texttt{cosh}, ale počítá hyperbolický tangens přes exponenciální funkci. Tato
  funkce je rychle rostoucí, již pro malé argumenty nabývá velkých hodnot. Obecně
  by se dala řešit pomocí Taylorovy řady, která ale
  pro velká číslo konverguje příliš pomalu. Nejrychlejší konvergence
  je pro čísla menší jak jedna. Proto je samotný výpočet exponeniální funkce rozdělen do
  dvou částí. Na část celočíselnou a zlomkovou.

  Argument exponenciální funkce je rozdělen na dvě části. Mějme například číslo $3.5$,
  které rozdělíme na $3$ a $0.5$. Funkce \texttt{approx\_exp()}, která bere dva argumenty --
  číslo \texttt{x} a uživatelem zadanou přesnost výpočtu $\varepsilon$ -- postupně volá
  dvě funkce \texttt{approx\_by\_int\_exp()} a \texttt{approx\_by\_dbl\_exp()}. Prvně
  jmenovaná funkce počítá celočíselnou část argumentu a druhá desetinnou. Oba
  výsledky se pak vynásobí a vrátí uživateli. Výpočet využívá rovnosti:

  \[
  e^{a + b} = e^a \cdot e^b
  \]

  Funkce \texttt{approx\_by\_dbl\_exp} implementuje Taylorovu řadu. Jelikož ji voláme
  pouze pro čísla menší než jedna a větší než nula, podává uspokojivé výsledky při
  malém počtu iterací. Druhá funkce, \texttt{approx\_by\_int\_exp} pracuje s celými
  čísly. Pro příklad čísla $3$ funkce provede dvě iterace. Číslo $3$ se dá ve
  dvojkové soustavě zapsat jako $0011$ -- tedy součet čísel $1$ a $2$ a tedy jako:

  \[
  e^{3} = e^{1} \cdot e^{2}
  \]

  Funkce v každé iteraci násobí exponent mocnin dvojky k průběžnému výsledku. Počet nutných
  iterací je roven nejvyššímu bitu argumentu exponenciální funkce. Pro číslo $8$ jsou
  to $4$ iterace, pro číslo $32$ jen $6$, atd\ldots\newline

  \noindent
  Výpočet \texttt{tanh} závisí na rychlosti a přesnoti implementace exponenciální
  funkce. Jelikož je funkce lichá, není potřeba upravovat algoritmus pro záporné
  argumenty, stačí jenom výsledku změnit znaménko. Také se při výpočtu využívá faktu,
  že funkce se velmi rychle přiblíží číslu jedna. Stačí tedy na začátku zjistit maximální
  hodnotu argumentu \texttt{tanh}, pro který již nestačí přesnost datového typu
  \texttt{double}. To vypočítáme přes funkci \texttt{artanh}:

  \[
  x_{max} = artanh \; \varepsilon_{dbl} = \frac{1}{2} \ln \left( \frac{1 + \varepsilon_{dbl}}{1 - \varepsilon_{dbl}} \right)
  \]

  kde $x_{max}$ je hledaná maximální hodnota, kterou jsme ještě schopni vyčíslit pomocí
  \texttt{double} a $\varepsilon_{dbl}$ je konstanta \texttt{DBL\_EPSILON} z knihovny
  \texttt{float.h}.\newline

  \noindent
  Tato maximální hodnota se vypočítá jednou, při spuštění programu a ukládá se do
  \texttt{IZP\_TANH\_MAX}. Pro argumenty $x > IZP\_TANH\_MAX$ se proto vrací výsledek
  $1$ bez dalšího výpočtu.

%=============================================================================
\subsection{Implementace \texttt{logax}} \label{kap-logax}

Funkce obecného logaritmu se volá pomocí \texttt{approx\_log\_a()}. Zde
se rozhodne, jestli je argument nebo základ logaritmu menší jak jedna. Pokud ano,
vypočítá se převrácená hodnota čísla a výsledek se pak vrací s opačným znaménkem.
Tento poznatek, popsaný v kapitole \ref{kap-log-mat}, jsem využil při návrhu
algoritmu pro výpočet logaritmu -- očekávám vstupní hodnoty pouze větší jak $1$.

Podobně jako u hyperbolického tangens i zde je samotný výpočet rozdělen do dvou
částí -- na celočíselný a desetinný výsledek. Výsledky jsou pak sečteny
a vráceny uživateli. Toto je implementováno ve funkcích \texttt{integer\_log\_a1\_x1()}
a \texttt{fractional\_log\_a1\_x1()}.\newline

\noindent
Princip výpočtu si ukážeme na funkci \texttt{integer\_log\_a1\_x1()}. V následujícím
textu předpokládám toto:

\[
y = \log_a x; \;\;\;\; a^y = x
\]

Funkce bere čísla \texttt{a} a \texttt{x} jako své argumenty. \texttt{y} je výsledek,
který je předáván jako návratová hodnota. Základní myšlenka algoritmu je projít všechny
bity čísla \texttt{y} a zjistit jestli jsou nastavené, nebo ne. Toto probíhá v
jednom cyklu, kdy argument \texttt{x} dělím číslem $a^n$ (\texttt{n} je n-tý bit v
\texttt{y}). Pokud je výsledek dělení větší jak jedna, je daný bit čísla \texttt{y}
nastaven. Můžeme tak napsat:

\[
x_n = \frac{x_{n-1}}{a^n}
\]

Jestli je \texttt{n}-tý bit výsledku \texttt{y} nastaven, je v \texttt{n}-té iteraci
\texttt{$x_n$} větší jak jedna. Číslo \texttt{y} inkrementujeme o jedničku, posuneme
jeho bity doleva a pokračujeme další iterací. Jakmile takto určíme všechny bity čísla \texttt{y},
máme hledaný vysledek uložen v \texttt{y} a zbytek v \texttt{x}. Tento zbytek je
dále předán druhé funkci -- \texttt{fractional\_log\_a1\_x1()} -- která je principiálně
stejná jako první, ale počítá logaritmus čísla \texttt{x} menší jak základ \texttt{a}.

%=============================================================================
\subsection{Implementace \texttt{wam} a \texttt{wqm}} \label{kap-stat}

Statistické funkce jsou implementovány ve funkcích \texttt{calc\_wam} -- vážený
aritmetický průměr -- a \texttt{calc\_wqm} -- vážený kvadratický průměr. Obě funkce
berou jako první parametr ukazatel na strukturu \texttt{s\_mean} do které se ukládají
průběžné výsledky, druhým a třetím parametrem jsou čísla typu \texttt{double}, které
představují hodnotu, resp. váhu \texttt{n}-tého prvku. Návratová hodnota obou funkcí
je průběžný výsledek.

%%%%%%%%%%%%%%%%%%%%%%%%%%%%%%%%%%%%%%%%%%%%%%%%%%%%%%%%%%%%%%%%%%%%%%%%%%%%%%
%%%%%%%%%%%%%%%%%%%%%%%%%%%%%%%%%%%%%%%%%%%%%%%%%%%%%%%%%%%%%%%%%%%%%%%%%%%%%%


%%%%%%%%%%%%%%%%%%%%%%%%%%%%%%%%%%%%%%%%%%%%%%%%%%%%%%%%%%%%%%%%%%%%%%%%%%%%%%
% seznam citované literatury: každá položka je definována příkazem
% \bibitem{xyz}, kde xyz je identifikátor citace (v textu použij: \cite{xyz})
\begin{thebibliography}{1}

% jedna citace:
\bibitem{kalendar}
KNUTH E. Donald: \emph{The Art Of Computer Programming, Volume 2}.
Addison-Wesley, 1997, ISBN 0-20-189684-2.

\end{thebibliography}
%%%%%%%%%%%%%%%%%%%%%%%%%%%%%%%%%%%%%%%%%%%%%%%%%%%%%%%%%%%%%%%%%%%%%%%%%%%%%%
% přílohy
\appendix

\section{Metriky kódu} \label{metriky}
\paragraph{Počet souborů:} 1 soubor
\paragraph{Počet řádků zdrojového textu:} 632 řádků
\paragraph{Velikost statických dat:} 15361B
\paragraph{Velikost spustitelného souboru:} 14544B (systém Darwin 10.5.0, 64 bitová
architektura, gcc 4.2.1, při překla\-du bez ladicích informací)


\end{document}
